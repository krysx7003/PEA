\documentclass{article}
\usepackage{graphicx}
\usepackage{float}
\usepackage{titlesec}
\usepackage{datetime}
\usepackage{geometry}
\usepackage{placeins}
\usepackage{minted}
\usepackage{xcolor}
\usepackage{listings}
\usepackage{caption}
\usepackage[document]{ragged2e}
\usepackage[hidelinks]{hyperref}
\usepackage{enumitem}
\geometry{
 a4paper,
 left=25mm,
 top=25mm,
 }
\captionsetup{hypcap=false} 
\newdateformat{daymonthyear}{\THEDAY .\THEMONTH .\THEYEAR}
\title{
  \centering
  \includegraphics[width=\textwidth]{src/images/logo_PWr_kolor_poziom.png}\\
  \fontsize{28pt}{30pt}\selectfont Programowanie efektywnych algorytmów\\
  \fontsize{14pt}{30pt}\selectfont Problem komiwojażera (TSP)}
\author{Krzysztof Zalewa}
\date{\daymonthyear\today}
\renewcommand*\contentsname{Spis treści}
\renewcommand{\figurename}{Rysunek}
\renewcommand{\listingscaption}{Fragment kodu}
\begin{document}
    \maketitle
    \pagebreak
    \tableofcontents
    \FloatBarrier
    \section{Wstęp teoretyczny}
      \begin{figure}[ht]
        \centering
        \includegraphics[width=\textwidth]{src/images/logo_PWr_kolor_poziom.png}
        \caption{}
        \label{fig:tex2}
      \end{figure}
      \FloatBarrier
    \subsection{Tabu search}
      \subsubsection{Swap lub Insert} 
        (4*ASYM + 4*ASM)*2 * 4 64
      \subsubsection{NN lub random}
        (4*ASYM + 4*ASM)*2 * 4 64
      \subsubsection{Iteracje bez zmian}
        (4*ASYM + 4*ASM)*5 * 4 160
      \subsubsection{Długość tabu} 
        (4*ASYM + 4*ASM)*5 * 4 160
      \subsubsection{Podsumowanie}ok 5h
    \subsection{Simulated anealing}
      \subsubsection{Swap lub Insert}
        (4*ASYM + 4*ASM)*2 * 5 
      \subsubsection{NN lub random}
        (4*ASYM + 4*ASM)*2 * 5 
      \subsubsection{Długość epoki}
        (4*ASYM + 4*ASM)*5 * 4 
      \subsubsection{Wielkość alfa}
        (4*ASYM + 4*ASM)*5 * 4 
      \subsubsection{Temperatura startowa}
        (4*ASYM + 4*ASM)*5 * 4 1
      \subsubsection{Podsumowanie}ok 8,5h
    \subsection{Algorytm mrówkowy}
      \subsubsection{Rozkład feromonów}
        (4*ASYM + 4*ASM)*2 * 4 64
      \subsubsection{Wartość rho}
        (4*ASYM + 4*ASM)*5 * 4 160
      \subsubsection{Stosunek alfa do bety}
        (4*ASYM + 4*ASM)*5 * 4 160
      \subsubsection{Podsumowanie}ok 7h
    \section{Zadanie laboratoryjne}


    \section{Wnioski}


    \section{Źródła}
      \begin{enumerate}[label=\arabic*.]
        \item \url{https://www.javatpoint.com/what-is-a-tabu-search}
        \item \url{https://www.geeksforgeeks.org/what-is-tabu-search/}
        \item \url{https://www.baeldung.com/cs/tabu-search}
      \end{enumerate}
\end{document}